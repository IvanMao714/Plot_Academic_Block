\documentclass[tikz]{standalone}
% This line declares a new math operator

% This line defines a new command for the differential operator.
\newcommand{\diff}{\mathop{}\!\mathrm{d}}

% This line loads the positioning library of TikZ.
\usetikzlibrary{positioning}

% This block defines some custom styles for the nodes and edges in the diagram.
\tikzset{
    % The hiddennode style is used for nodes that are not directly visible.
    hiddennode/.style={draw, circle, fill=green!20, text centered},

    % The signal style is used for edges that represent signals.
    signal/.style={->, >=latex, line width=1.5pt, draw=black}
}
\begin{document}
    \begin{tikzpicture}[]
    \def\pindist{35pt}
    \def\nodesize{38pt}
    \tikzstyle{every pin edge}=[signal]
    \tikzstyle{annot} = [text width=4em, text centered]
    \node[hiddennode, text width=\nodesize, minimum size=\nodesize,
        pin={[pin edge={latex-}, pin distance=\pindist]above left:$x_1$},
        pin={[pin edge={latex-}, pin distance=\pindist]below left:$x_2$},
        pin={[pin edge={-latex}, pin distance=\pindist]right:$y_1$}
        ] (N1) at (-120pt,0) {$f(x)$};

    \node[hiddennode, text width=\nodesize, minimum size=\nodesize,
        pin={[pin edge={-latex}, pin distance=\pindist]above left:$\frac{\partial L}{\partial x_1}=\frac{\partial L}{\partial y_1}\frac{\partial y_1}{\partial x_1}$},
        pin={[pin edge={-latex}, pin distance=\pindist]below left:$\frac{\partial L}{\partial x_2}=\frac{\partial L}{\partial y_1}\frac{\partial y_1}{\partial x_2}$},
        pin={[pin edge={latex-}, pin distance=\pindist]right:$\frac{\partial L}{\partial y_1}$}
        ] (N2) at (+120pt,0) {$\diff f$};

    % These blocks define some annotations for the diagram.
    \node[annot, text width=200pt, align=center, above=40pt of N1] (l1) {Forwardpass};
    \node[annot, text width=200pt, align=center, above=40pt of N2] (l2) {Backwardpass};

    \draw[signal, -] (0,-70pt) -- (0,+80pt);
\end{tikzpicture}
\end{document}
            